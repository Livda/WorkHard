\documentclass{rapport}
\author{Aurélien Fontaine \\ Damien Duvacher \\ Florentin Hortet}
\title{Compilation : Compte Rendu TP2}
\begin{document}
  \maketitle

  \chapter{Préparation du TP}
  \label{chap:PrepTP}
  Voici la grammaire obtenue après avoir supprimé les règles ne conscernant
  pas l'annalyse syntaxique, et mise sous forme LL(1) :
  \verbatimtabinput[3]{../g1}
  Pour les prédicats \textit{null}, \textit{premier} et \textit{suivant}
  de cette grammaire, on obtient :
  \verbatimtabinput[3]{predic}

  \chapter{Analyseur LL1}
  \label{chap:AnalyseurLL1}
  La fonction \textit{is\_LL1} n'est pas totalement fonctionnelle,
  elle reconnait trop de grammaires.
  La fonction \textit{deriv} n'a pas été implémentée du tout.
  Voici notre code de \textit{lL1.ml} :
  \verbatimtabinput[3]{../lL1.ml}

  \chapter{Questions de compréhention du TP}
  \label{chap:QuestionsTP}
  \section{Question 2.1}
  \textit{Pourquoi n'inclut-on pas les commentaires dans la grammaire
  définissant les constructions du langage ?}

  Lors de l'analyse lexicale, les commentaires sont supprimés car ils
  ne sont pas utiles pour le compilateur. L'analyse syntaxique n'a donc pas
  besoin de connaitre les commentaires, et donc la grammaire non plus.

  \section{Question 2.2}
  \textit{Pourquoi utilise-t-on des grammaires \emph{LL(1)} ? Quel est
  l'intérêt d'une grammaire \emph{LL(1)} ?}

  Les grammaires LL(1) sont des grammaires non ambigues, sans récurtion à
  gauche. Et les propriétés entre deux productions permettent de construire
  un analyseur prédictif en ne regardant qu'un élément suivant.

  \section{Question 2.3}
  \textit{Comment fait-on le lien entre analyse lexicale et analyse syntaxique
  ?}

  Ici le lien entre les deux analyseurs est un arbre.

  \chapter{Tests}
  \label{chap:Tests}
  \section{Grammaire proposée}
  Nous avons réalisé un test avec la grammaire \emph{g3}, proposée par défaut :
  \verbatiminput{../g3}

  Nous optenons :
  \verbatiminput{solg3}


  La grammaire n'est pas LL(1) mais les prédicats sont bons. On retrouve les
  mêmes résulats avec un autre grammaire dont deux prédicats permettent
  d'atteindre l'état \emph{null}. Voici la grammaire \emph{g2} :
  \verbatiminput{../g2}

  Et voici le résulat :
  \verbatiminput{solg2}

  Ici encore, la grammaire n'est pas LL(1) mais les prédicats sont bons.


\end{document}
