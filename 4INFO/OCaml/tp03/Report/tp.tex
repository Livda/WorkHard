\documentclass{rapport}
\author{Aurélien Fontaine\\Damien Duvacher}
\title{Compilation : Compte Rendu TP3}
\begin{document}

    \maketitle

    \chapter{Préparation du TP}
    Après avoir supprimé les règles 15 et 16 car elles ne concernent pas
    l'analyse syntaxique, nous obtenons la grammaire suivante à partir de
    la ligne 9:

    \verbatiminput{grammar}

    A partir de cette grammaire nous avons fait sa clôture SLR
    (voir chapitre \ref{slr_closure} page \pageref{slr_closure}).
    Puis nous avons obtenu cette table SLR en faisant attention aux conflits :
    \begin{table}[!ht]
  \centering
  \begin{tabular}{|c||c|c|c|c|c|c|c|c||c|c|}
    \hline
    \multirow{2}{*}{State} & \multicolumn{8}{c||}{Action} & \multicolumn{2}{c|}{GOTO}\\
    \cline{2-11} & id & $\leftarrow$ & + & $<$ & and & ( & ) &\$ & I & E\\
    \hline
    0  &s1 &   &   &   &   &   &   &   &   &\\
    \hline
    1  &   &s2 &   &   &   &   &   &   &   &\\
    \hline
    2  &s5 &   &   &   &   &s4 &   &   &   &3\\
    \hline
    3  &   &   &s6 &s7 &s8 &   &   &acc&   &\\
    \hline
    4  &s5 &   &   &   &   &s4 &   &   &   &9\\
    \hline
    5  &   &   &r5 &r5 &r5 &   &r5 &r5 &   &\\
    \hline
    6   &s5 &   &   &   &   &s4 &   &   &   &10\\
    \hline
    7   &s5 &   &   &   &   &s4 &   &   &   &11\\
    \hline
    8   &s5 &   &   &   &   &s4 &   &   &   &12\\
    \hline
    9   &   &   &s6 &s7 &s8 &   &s13&   &   &\\
    \hline
    10  &   &   &r1 &r1 &r1 &   &r1 &r1 &   &\\
    \hline
    11  &   &   &s6 &r2 &r2 &   &r2 &r2 &   &\\
    \hline
    12  &   &   &s6 &s7 &s8 &   &r3 &r3 &   &\\
    \hline
    13  &   &   &r4 &r4 &r4 &   &r4 &r4 &   &\\
    \hline
  \end{tabular}
  \caption{SLR table}
\end{table}


    \chapter{Code du TP}-

    Voici les différents fichiers que nous avons développés pour ce TP.

    \section{parser.ml}
    \verbatiminput{../parser.mly}

    \section{lexer.mll}
    \verbatiminput{../lexer.mll}

    \section{main.ml}
    \verbatiminput{../main.ml}

    \section{type.ml}
    \verbatiminput{../type.ml}

    \chapter{Tests effectués}

    \section{Tests validé}
    \begin{multicols}{2}
    Voici le jeu de test avec tous les tokens d'utilisés :
    \verbatiminput{../test1}
    \vspace{5pt}
    \setlength{\columnseprule}{0.5pt}
    Qui a donné les résultats suivants :
    \verbatiminput{../res1}
    \end{multicols}

    \section{Tests sur les balises BEGIN et END}
    \begin{multicols}{2}
        Voici le jeu de test pour la balise  "begin" :
        \verbatiminput{../test2}

        Qui a donné les résultats suivants :
        \verbatiminput{../res2}

        \setlength{\columnseprule}{0.5pt}

        Voici le jeu de test pour la balise "end" :
        \verbatiminput{../test3}

        Qui a donné les résultats suivants :
        \verbatiminput{../res3}
    \end{multicols}

    \newpage
    \section{Tests sur les types}
    \begin{multicols}{2}
        Voici le jeu de test pour les entiers :
        \verbatiminput{../test4}
        Qui a donné les résultats suivants :
        \verbatiminput{../res4}
        \setlength{\columnseprule}{0.5pt}
        Voici le jeu de test pour les booléens :
        \verbatiminput{../test5}
        Qui a donné les résultats suivants :
        \verbatiminput{../res5}
    \end{multicols}

    \section{Tests sur la ponctuation}
    \begin{multicols}{2}
        Voici le premier jeu de test sur la ponctuation dans les déclarations :
        \verbatiminput{../test6}
        Qui a donné les résultats suivants :
        \verbatiminput{../res6}
        Voici le second jeu de test sur la ponctuation dans les déclarations :
        \verbatiminput{../test7}
        Qui a donné les résultats suivants :
        \verbatiminput{../res7}
        \setlength{\columnseprule}{0.5pt}
        Voici le premier jeu de test sur la ponctuation dans les instructions :
        \verbatiminput{../test8}
        Qui a donné les résultats suivants :
        \verbatiminput{../res8}
        Voici le second jeu de test sur la ponctuation dans les instructions :
        \verbatiminput{../test9}
        Qui a donné les résultats suivants :
        \verbatiminput{../res9}
    \end{multicols}

    \section{Test sur les affectations}
    Voici le jeu de test pour les affectations :
    \verbatiminput{../test10}
    Qui a donné les résultats suivants :
    \verbatiminput{../res10}

    \appendix
    \chapter{SLR Closure Table}
    \label{slr_closure}
    \begin{table}[!ht]
  \hspace{-1.4cm}
  \begin{minipage}[t]{.4\linewidth}
    \centering
    \begin{tabular}{|c|c|c|c|}
      \hline
      GOTO & Kernel & State & Closure\\
      \hline
      &	\{I $\Rightarrow$ .id $\leftarrow$ E\} & 0 & \{I $\Rightarrow$ .id $\leftarrow$ E\}\\
      \hline
      goto(0, id)	& \{I $\Rightarrow$ id.$\leftarrow$ E\} &	1	&	\{I $\Rightarrow$ id.$\leftarrow$ E\}\\
      \hline
      \multirow{6}{*}{goto(1, $\leftarrow$)} & \multirow{6}{*}{\{I $\Rightarrow$ id $\leftarrow$.E\}} & \multirow{6}{*}{2} & \{I $\Rightarrow$ id $\leftarrow$.E;\\
      & & & E $\Rightarrow$ .E + E;\\
      & & & E $\Rightarrow$ .E $<$ E;\\
      & & & E $\Rightarrow$ .E and E;\\
      & & & E $\Rightarrow$ .( E );\\
      & & & E $\Rightarrow$ .id\}\\
      \hline
      \multirow{4}{*}{goto(2, E)} & \{I $\Rightarrow$ id $\leftarrow$ E.; & \multirow{4}{*}{3} & \{I $\Rightarrow$ id $\leftarrow$ E.;\\
      & E $\Rightarrow$ E.+ E; & & E $\Rightarrow$ E.+ E; \\
      & E $\Rightarrow$ E.$<$ E; & & E $\Rightarrow$ E.$<$ E; \\
      & E $\Rightarrow$ E.and E\} & & E $\Rightarrow$ E.and E\} \\
      \hline
      \multirow{6}{*}{goto(2, ()} & \multirow{6}{*}{\{E $\Rightarrow$ (.E )\}} & \multirow{6}{*}{4}	& \{E $\Rightarrow$ (.E );\\
      & & & E $\Rightarrow$ .E + E;\\
      & & & E $\Rightarrow$ .E $<$ E;\\
      & & & E $\Rightarrow$ .E and E; \\
      & & & E $\Rightarrow$ .( E );\\
      & & & E $\Rightarrow$ .id\}\\
      \hline
      goto(2, id) & \{E $\Rightarrow$ id.\} & 5 & \{E $\Rightarrow$ id.\}\\
      \hline
      \multirow{5}{*}{goto(3, +)} & \multirow{5}{*}{\{E $\Rightarrow$ E +.E\}}	& \multirow{5}{*}{6}	& \{E $\Rightarrow$ E +.E; \\
      & & & E $\Rightarrow$ .E + E; \\
      & & & E $\Rightarrow$ .E $<$ E; \\
      & & & E $\Rightarrow$ .E and E; \\
      & & & E $\Rightarrow$ .( E ); \\
      & & & E $\Rightarrow$ .id\}\\
      \hline
      \multirow{6}{*}{goto(3, $<$)} & \multirow{6}{*}{\{E $\Rightarrow$ E $<$.E\}}	& \multirow{6}{*}{7}	& \{E $\Rightarrow$ E $<$.E; \\
      & & & E $\Rightarrow$ .E + E; \\
      & & & E $\Rightarrow$ .E $<$ E; \\
      & & & E $\Rightarrow$ .E and E; \\
      & & & E $\Rightarrow$ .( E ); \\
      & & & E $\Rightarrow$ .id\}\\
      \hline
      \multirow{6}{*}{goto(3, and)} & \multirow{6}{*}{\{E $\Rightarrow$ E and.E\}}	& \multirow{6}{*}{8}	& \{E $\Rightarrow$ E and.E; \\
      & & & E $\Rightarrow$ .E + E; \\
      & & & E $\Rightarrow$ .E $<$ E; \\
      & & & E $\Rightarrow$ .E and E; \\
      & & & E $\Rightarrow$ .( E ); \\
      & & & E $\Rightarrow$ .id\}\\
      \hline
    \end{tabular}
  \end{minipage}
  \hspace{3cm}
  \begin{minipage}[t]{.4\linewidth}
    \begin{tabular}{|c|c|c|c|}
      \hline
      GOTO & Kernel & State & Closure\\
      \hline
      \multirow{4}{*}{goto(4, E)} & \{E $\Rightarrow$ ( E.); & \multirow{4}{*}{9}	& \{E $\Rightarrow$ ( E.); \\
      & E $\Rightarrow$ E.+ E; & & E $\Rightarrow$ E.+ E; \\
      & E $\Rightarrow$ E.$<$ E; & & E $\Rightarrow$ E.$<$ E; \\
      & E $\Rightarrow$ E.and E\} & & E $\Rightarrow$ E.and E\}\\
      \hline
      goto(4, ()	& \{E $\Rightarrow$ (.E )\}	& 4	& \\
      \hline
      goto(4, id)	& \{E $\Rightarrow$ id.\} & 5 & \\
      \hline
      \multirow{4}{*}{goto(6, E)} & \{E $\Rightarrow$ E + E.; & \multirow{4}{*}{10}	& \{E $\Rightarrow$ E + E.; \\
      & E $\Rightarrow$ E.+ E; & & E $\Rightarrow$ E.+ E; \\
      & E $\Rightarrow$ E.$<$ E; & & E $\Rightarrow$ E.$<$ E; \\
      & E $\Rightarrow$ E.and E\} & & E $\Rightarrow$ E.and E\} \\
      \hline
      goto(6, () & \{E $\Rightarrow$ (.E )\}	& 4 & \\
      \hline
      goto(6, id)	& \{E $\Rightarrow$ id.\} & 5 & \\
      \hline
      \multirow{4}{*}{goto(7, E)} & \{E $\Rightarrow$ E $<$ E.; & \multirow{4}{*}{11} & \{E $\Rightarrow$ E $<$ E.; \\
      & E $\Rightarrow$ E.+ E; & & E $\Rightarrow$ E.+ E; \\
      & E $\Rightarrow$ E.$<$ E; & & E $\Rightarrow$ E.$<$ E; \\
      & E $\Rightarrow$ E.and E\}	& & E $\Rightarrow$ E.and E\} \\
      \hline
      goto(7, () & \{E $\Rightarrow$ (.E )\}	& 4 & \\
      \hline
      goto(7, id)	& \{E $\Rightarrow$ id.\} & 5 & \\
      \hline
      \multirow{4}{*}{goto(8, E)} & \{E $\Rightarrow$ E and E.; & \multirow{4}{*}{12} & \{E $\Rightarrow$ E and E.; \\
      & E $\Rightarrow$ E.+ E; & & E $\Rightarrow$ E.+ E; \\
      & E $\Rightarrow$ E.$<$ E; & & E $\Rightarrow$ E.$<$ E; \\
      & E $\Rightarrow$ E.and E\}	& & E $\Rightarrow$ E.and E\} \\
      \hline
      goto(8, () & \{E $\Rightarrow$ (.E )\}	& 4 & \\
      \hline
      goto(8, id)	& \{E $\Rightarrow$ id.\} & 5 & \\
      \hline
      goto(9, )) & \{E $\Rightarrow$ ( E ).\} & 13	& \{E $\Rightarrow$ ( E ).\} \\
      \hline
      goto(9, +) & \{E $\Rightarrow$ E +.E\}	& 6 & \\
      \hline
      goto(9, $<$) & \{E $\Rightarrow$ E $<$.E\}	& 7 & \\
      \hline
      goto(9, and) & \{E $\Rightarrow$ E and.E\}	& 8 & \\
      \hline
      goto(10, +)	& \{E $\Rightarrow$ E +.E\} & 6 & \\
      \hline
      goto(10, $<$)	& \{E $\Rightarrow$ E $<$.E\} & 7 & \\
      \hline
      goto(10, and)	& \{E $\Rightarrow$ E and.E\} & 8 & \\
      \hline
      goto(11, +)	& \{E $\Rightarrow$ E +.E\} & 6 & \\
      \hline
      goto(11, $<$)	& \{E $\Rightarrow$ E $<$.E\} & 7 & \\
      \hline
      goto(11, and)	& \{E $\Rightarrow$ E and.E\} & 8 & \\
      \hline
      goto(12, +)	& \{E $\Rightarrow$ E +.E\} & 6 & \\
      \hline
      goto(12, $<$)	& \{E $\Rightarrow$ E $<$.E\} & 7 & \\
      \hline
      goto(12, and)	& \{E $\Rightarrow$ E and.E\} & 8 & \\
      \hline
    \end{tabular}
  \end{minipage}
  \caption{SLR closure table}
\end{table}



\end{document}
