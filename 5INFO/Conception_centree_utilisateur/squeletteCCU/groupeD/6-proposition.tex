%%% ATTENTION, le format demand� ici est diff�rent de celui du rapport
%%% 2009-10 !

\section{Propositions du  groupe X}

Nous rappelons que la cat�gorie d'utilisateurs cibl�e par le groupe X
concerne les ...

%%% ATTENTION, TOUTES LES PROPOSITIONS DOIVENT �TRE JUSTIFI�ES PAR
%%% RAPPORT AUX SC�NARIOS OBSERV�S. DE PLUS, ELLES DOIVENT �TRE
%%% JUSTIFI�ES EN D�TAIL.

%% Les modifications doivent �tre expliqu�es de mani�re � ce qu'on
%% croit que �a puisse �tre impl�ment�.

%%% Dupliquer la "subsection" suivante autant de fois que
%%% n�cessaire.

%%% Ins�rer autant de figures que n�cessaires pour illustrer les
%%% am�liorations. Il vaut mieux qu'il y en ait trop que pas
%%% assez. Soyez g�n�reux !

%%% Penser � utiliser des labels qui contiennent le nom du groupe afin
%%% qu'il n'y ait pas de clash de nom au moment de l'int�gration
%%% globale du rapport, par exemple \label{groupeX:fig:menu:insertion}.

%%% Utiliser syst�matiquement des r�f�rences explicites aux items de
%%% la partie sur les sc�narios d'utilisation.

\subsection{XXX}  %  XXX = Nom d'une fonctionnalit� ou d'un th�me ou
                  %  d'un aspect ou...

\begin{enumerate}
\item {\em XXX : } YYY.	%  XXX = Nom  d'un aspect, YYY = qqs d�tails

{\em Discussion.} Nous pensons que cette am�lioration permettra
d'�viter le probl�me ZZZZ rencontr� par l'utilisateur lors de
l'entretien~\ref{groupeX:utilisation:entretienWWW} d�taill�
page~\pageref{groupeX:utilisation:entretienWWW}. En effet, ...
Cette proposition devrait �galement permettre de r�gler le probl�me....

\end{enumerate}


