\section{Audits du groupe X}
\label{groupeX:audit}
Nous rappelons que la cat�gorie d'utilisateurs cibl�e par le groupe X
concerne les ...

%%% Les entretiens doivent �tre dans le M�ME ORDRE que pour les
%%% sc�narios d'utilisation.



%%% Il doit y avoir au moins autant d'entretiens que d'�tudiants dans
%%% le groupe.

%%% Dupliquer la "subsection" suivante autant de fois que n�cessaire.

%%% Le rapport doit �tre anonyme en ce qui concerne les personnes
%%% interview�es et celles dans leur contexte. Par contre le nom des
%%% �tudiants intervenants doit �tre donn�.

\subsection{Entretien avec XXX }
%%% XXX : reprendre le m�me profil qu'en 1-utilisation et l'actualiser
%%% si n�cessaire

\begin{description}
\item [Intervieweur : ] % le nom complet mais pas de nom tout en majuscule 
\item [Secr�taire : ] % idem
\item [Cameraman/photographe : ] % idem
\item [Observateur : ] % idem
\item [Personne interrog�e, sa fonction  et son �ge: ] %%% Ses initiales seulement
\item [Lieu : ]
\item [Date et dur�e : ]
\item [Contexte : ] %% ATTENTION de l'actualiser. C'est souvent un peu
                    %% diff�rent du premier entretien.
\end{description}
\begin{enumerate}
\item %% ...  mettre les demandes et intervention de
      %% l'intervieweur en italique \emph{La demande en question}
\item %% ...
\end{enumerate}
